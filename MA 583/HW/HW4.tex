%%%%% Beginning of preamble %%%%%

\documentclass[12pt]{article}  %What kind of document (article) and what size
\usepackage[document]{ragged2e}

\usepackage{wrapfig}
%Packages to load which give you useful commands
\usepackage{graphicx}
\usepackage{amssymb, amsmath, amsthm}
\usepackage{fancyhdr}
\usepackage[linguistics]{forest}
\usepackage{enumerate}
\usepackage{blkarray}
\usepackage[margin=1in]{geometry} 
\pagestyle{fancy}
\fancyhf{}
\lhead{MA 583: HW4, \today, Discussion A4}
\rhead{Benjamin Draves}


\renewcommand{\headrulewidth}{.4pt}
\renewcommand{\footrulewidth}{0.4pt}

\topmargin = -0.4 in
%\headheight = 0.0 in t
%\headsep = .3 in
\parskip = 0.2in
%\parindent = 0.0in

%%%%%%%%%%new commands%%%%%%%%%%%%
\newcommand{\N}{{\mathbb{N}}}
\newcommand{\Z}{{\mathbb{Z}}}
\newcommand{\R}{{\mathbb{R}}}
\newcommand{\Q}{{\mathbb{Q}}}
\newcommand{\e}{{\epsilon}}
\newcommand{\del}{{\delta}}
\newcommand{\m}{{\mid}}
\newcommand{\infsum}{{\sum_{n=1}^\infty}}
\newcommand{\la}{{\langle}}
\newcommand{\ra}{{\rangle}}
\newcommand{\E}{{\mathbb{E}}}
\newcommand{\V}{{\text{Var}}}
\newcommand{\prob}{{\mathbb{P}}}
\newcommand{\ind}{{\mathbf{1}}}

%defines a few theorem-type environments
\newtheorem{theorem}{Theorem}
\newtheorem{corollary}[theorem]{Corollary}
\newtheorem{definition}{Definition}
\newtheorem{lemma}[theorem]{Lemma}
%%%%% End of preamble %%%%%

\begin{document}

\begin{description}
\item[Exercise 3.5.2]
\begin{enumerate}[(a)]
\item Let $A_n$ be the wealth of player $A$ at time $n$. Then we can write $A_n = A_{n-1} + 1$ with probability $p$ and $A_n = A_{n-1}-1$ with probability $1-p$. We recognize the Gambler's ruin game as a random walk on $\{0, 1, \ldots, 100\}$ with absorbing boundaries. From equation (3.4.2) we can calculate the ruin probabilities as follows; let $u_i = \prob(\text{$A_n$ reaches state 0 before state $100$}|A_0 = i)$. Then in our case (as $p\neq 1 - p$) we have $$u_i = \frac{(\frac{1-p}{p})^i - (\frac{1-p}{p})^{100}}{1 - (\frac{1-p}{p})^{100}}$$
Now, for $i = 50$ and $p = 0.49292929$ we have 
$$u_{50} = \frac{(\frac{1-0.49292929}{0.49292929})^{50} - (\frac{1-0.49292929}{0.49292929})^{100}}{1 - (\frac{1-0.49292929}{0.49292929})^{100}} = 0.804433$$
For $i = 500$ then our random walk is now over $\{0, 1, \ldots , 1000\}$ so we have $n = 1000$ and our equation becomes
$$u_i = \frac{(\frac{1-p}{p})^{i} - (\frac{1-p}{p})^{1000}}{1 - (\frac{1-p}{p})^{1000}}$$
In which case for $i = 500$ we have 
$$u_{500} = \frac{(\frac{1-0.49292929}{0.49292929})^{500} - (\frac{1-0.49292929}{0.49292929})^{1000}}{1 - (\frac{1-0.49292929}{0.49292929})^{1000}} = 0.9999993$$
\item Following an identical process as above but with $p = 0.5029237$ we have 
$$u_{50} = \frac{(\frac{1-0.5029237}{0.5029237})^{50} - (\frac{1-0.5029237}{0.5029237})^{100}}{1 - (\frac{1-0.5029237}{0.5029237})^{100}} = 0.3578411$$
\end{enumerate}$$u_{500} = \frac{(\frac{1-0.5029237}{0.5029237})^{500} - (\frac{1-0.5029237}{0.5029237})^{1000}}{1 - (\frac{1-0.5029237}{0.5029237})^{1000}} = 0.002878892$$


\item[Exercise 3.6.4] Let $T = \min\{n\geq 0: X_n\in\{1,3\}\}$ and define $v_i = \E[T|X_0 = i]$. Here we have $v_0 = v_3 = 0$. Moreover we see that \begin{align*}&v_1 = 1 + 0.7v_2 \hspace{2em} v_2 = 1 +0.3v_1 \iff\\ 
&v_1 = 1 + 0.7(1 + 0.3v_1) \iff \frac{79}{100}v_1 = \frac{17}{10} \iff v_1 = \frac{170}{79} \approx 2.151899 
\end{align*}
Now, by the equation proceeding equation (3.5.4) we see that $$v_1 = \frac{1}{p(1-\theta)}\left[N\left(\frac{1 - \theta}{1-\theta^N}\right) - 1\right]$$ where $N = 3$, $p = 0.7$, $q = 0.3$, and $\theta = \frac{q}{p}= \frac{3}{7}$. With all this we see that 
$$v_1 = \frac{1}{7/10*(1 - 3/7)}\left[3\left(\frac{1 - 3/7}{1 - (3/7)^3}\right) - 1\right] = 2.151899$$

\item[Exercise 3.8.1]Let $X_n$ be the number of individuals in generation $n$ and let $\xi^{(n)}_i$ be the number of progeny of individual $i$ from generation $n$. By assumption, we have $\xi_{i}^{(n)} = 2$ with probability $1/2$ and $\xi_{i}^{(n)} = 0$ with probability $1/2$. From here we see that $\E(\xi_{i}^{(n)}) = 1$ and $\V(\xi_{i}^{(n)}) = \E[(\xi_{i}^{(n)})^2] - \E[\xi_{i}^{(n)}]^2 = 2 - 1 = 1$. Using this we can define the size of the $n+1$ generation as the random sum $$X_{n+1} = \sum_{i = 1}^{X_n}\xi_i^{(n)}$$ From here we see that 
\begin{align*}
\E[X_{n+1}] &= \E[X_n]\E[\xi_1^{(n)}] = \E[X_n] = \ldots = \E[X_0] = 1\\
\V(X_{n+1}) &= \E[X_n]\V(\xi_1^{(n)}) + \E[\xi_1^{(n)}]^2\V(X_n)\\
&= 1 + \V(X_n) = 2 + \V(X_{n-1})\\
&\vdots\\
&= (n+1)+\V(X_0)\\  
&= n+1
\end{align*}
where the last equality is due to the fact that $X_0 = 1$ always. Therefore, $\E[X_n] = 1$ and $\V(X_n) = n$. 


\item[Problem 3.5.2]
\begin{enumerate}[(a)]
\item We begin by interpreting the $X_n$ in terms of $T$. 
$$p_i = \prob(X_{n+1} = 0|X_n = i) = \prob(T = i + 1|T>i) = \frac{\prob(T = i+1)}{\sum_{n = i+1}^{\infty}\prob(T = n)} = \frac{a_{i+1}}{\sum_{n = i+1}^{\infty}a_n}$$
Moreover, seeing that $X_{n+1} \neq X_{n}$ $r_0 = 0$ which implies that $$q_i = 1 -\frac{a_{i+1}}{\sum_{n = i+1}^{\infty}a_n} $$
\item When we enforce a planned replacement policy, we see that $p_N = 1$ and $q_N = 0$. Now, for $0\leq i <N$, the process is unaffected by planned replacement policy. That is $T$ is independent of $N$. Hence, for $0\leq i <N$ the $q_i$ and $p_i$ are given in (a).
\end{enumerate}

\item[Problem 3.5.5]
\begin{enumerate}[(a)]
\item First note that $X_n$ is a random walk on $\{0, 1, 2,\ldots\}$ so $X_n\geq 0$ for all $n$. Now suppose that $X_0 = k<\infty$. Then $\E|X_n| = \E(X_n) \leq \E(X_0) + n = k + n <\infty$. That is, for each $X_n$, it has taken at most $n$ `steps to the right' which is still a finite value. Now, for second martingale property we have 
\begin{align*}
\E[X_{n+1}|X_n, \ldots, X_{0}] &= \E[1/2(X_{n-1} + 1) + 1/2(X_{n-1} - 1)|X_{n -1}, \ldots, X_0]\\
&= \E[1/2X_{n-1} + 1/2X_{n-1}]\\
&= \E[X_{n-1}]
\end{align*}
Having shown these properties we see that $X_n$ is a nonnegative martingale.
\item Applying the maximal inequality we have $$\prob(\underset{n\geq 0}{\max}X_n\geq N)\leq \frac{\E(X_0)}{N} = \frac{k}{N}$$ 
\end{enumerate}
As the right side of this inequality is free from $n$, we have a uniform bound of this quantity for all values in the martingale. 

\item[Problem 3.6.7]
Let $T = \min\{n\geq 0: X_n\in\{0,3\}\}$ and define $v_i = \E[T|X_0 = i]$. Then we have $v_0 = v_3 = 0$ and 
\begin{align*}&v_1 = 1 + 0.7v_2 \hspace{2em} v_2 = 1 +0.1v_1 \iff\\ 
&v_1 = 1 + 0.7(1 + 0.1v_1) \iff \frac{93}{100}v_1 = \frac{17}{10} \iff v_1 = \frac{170}{93} \approx 1.827957 
\end{align*}
Now using the results from equation (3.6.6) we have $$v_1 = \frac{\Phi_1 + \Phi_2}{1 + \rho_1 + \rho_2}$$ where $\rho_1 = q_1/p_1$, $\rho_2 = q_1q_2/p_1p_2$, and $\Phi_1 = \frac{\rho_1}{q_1}$, $\Phi_2 = \frac{\rho_2}{q_1} + \frac{\rho_2}{q_1\rho_1}$ Evaluating these quantities, we see that 

$$v_1 = \frac{1.428571 + 1.269841}{1 + 0.4285714 + 0.04761905} = 1.827957$$


\item[Problem 3.6.8]
Let $T  =\min\{n\geq 0: X_n = 3\}$ and define $u_i = \E[T|X_0 = i]$. Note that $u_3 = 0$. From a first step analysis, we arrive at the system given below 
\begin{align*}
u_0 &= 1 + \alpha u_0 + \beta u_2\\
u_1 &= 1 + \alpha u_0 \\
u_2 &= 1 + \alpha u_0 + \beta u_1\\
\end{align*}
First note that $$u_0 = 1 + \alpha u_0 + \beta u_2 \iff (1-\alpha)u_0= 1 + \beta u_2 \iff \beta u_0 = 1 + \beta u_2 \iff u_0 = 1/\beta + u_2$$ Moreover, we see that $$u_0 - 1/\beta = u_2 = 1 + \alpha u_0 + \beta u_1 \iff \beta u_0 = 1 + 1/\beta + \beta u_1\iff 1/\beta + 1/\beta^2 + u_1$$
Lastly writting $u_1$ in terms of $u_0$ we have $$u_0 = 1/\beta + 1/\beta^2 + 1 + \alpha u_0 \iff \beta u_0 = 1/\beta + 1/\beta^2 + 1 \iff u_0 = 1/\beta + 1/\beta^2+ 1/\beta^3$$
Therefore we see that $$u_0 = \sum_{k = 1}^3\frac{1}{\beta^k }$$

\item[Problem 3.8.3]
\begin{enumerate}[(a)]
\item Let $N$ be the number of children this family has and let $S_k$ be the sex of the $k$-th child. We will begin by conditioning on the first child's sex. 
\begin{align*}
\prob(N = k) &= \sum_{s = F,M}\prob(N = k-1|S_1 = s)\prob(S_1 = s)\\ &= \frac{1}{2}\prob(N = k-1|S_1 = M) + \frac{1}{2}\prob(N =k-1|S_1 = F) 
\end{align*}
Now, we note that $N|S_1 = F$ is degenerate $2$. That is $\prob(N = k-1|S_1 = F) = \ind_{\{k = 2\}}$. In a similar way, we see that $N|S_1 = M\sim \text{Geom}(1/2)$ where $\prob(N = k-1|S_1 = M) = (1/2)^{k-2}(1/2) = 1/2^{k-1}$. Putting this together, we see that 
\[
\prob(N  =k) = \begin{cases}
(1/2)^{k} + 1/2 & k = 2\\
(1/2)^{k} & k\geq 3\\
\end{cases} = \begin{cases}
3/4 & k = 2\\
(1/2)^{k} & k\geq 3\\
\end{cases}
\]
\item Let $B$ be the number of boys a family has. Then again, conditioning on the sex of the first child we have $$\prob(B = k) = \frac{1}{2}\prob(B = k -1|S_1 = M) + \frac{1}{2}\prob(B = k|S_1 = F)$$ Again notice that $B|S_1 = F$ is $Bern(1/2)$. That is $\prob(B = 1|S_1 = F) = 1/2$ and $\prob(B = 0|S_1 = F) = 1/2$ and $\prob(B = k|S_1 = F) = 0$ for all $k\geq 3$. Moreover, we note that $B|S_1 = M \sim\text{Geom}(1/2)$ so $\prob(B = k -1|S_1 = M) = (1-1/2)^{k-1}(1/2) = (1/2)^k$ for $k\geq 1$. Putting this together we arrive at the following 
\[
\prob(B  =k) = \begin{cases}
(1/2)^2 & k = 0\\
(1/2)^{k+1} + (1/2)^2 & k = 1\\
(1/2)^{k+1} & k\geq 2\\
\end{cases} = \begin{cases}
1/4 & k = 0\\
1/2 & k = 1\\
(1/2)^{k+1} & k\geq 2\\
\end{cases} 
\]
\end{enumerate}

\item[Problem 3.9.3] Here, we introduce terms to see this integral as a Gamma density with parameters $(k+\alpha, \frac{1}{1+\theta})$. Using this, we can derive the distribution as follows. 
\begin{align*}
p_k &= \int_0^{\infty}\pi(k|\lambda)f(\lambda)d\lambda = \frac{\theta^{\alpha}}{k!\Gamma(\alpha)}\int_{0}^{\infty}e^{-(1+\theta)\lambda}\lambda^{(k+\alpha)-1}d\lambda\\
&= \frac{\theta^{\alpha}\Gamma(k+\alpha)(\frac{1}{1+\theta})^{k+\alpha}}{k!\Gamma(\alpha)}\frac{1}{\Gamma(k+\alpha)(\frac{1}{1+\theta})^{k+\alpha}}\int_{0}^{\infty}e^{-\frac{\lambda}{1/(1+\theta)}}\lambda^{(k+\alpha)-1}d\lambda\\
&= \frac{\theta^{\alpha}\Gamma(k+\alpha)}{(1+\theta)^{k+\alpha}k!\Gamma(\alpha)} = \left(\frac{\theta}{1+\theta}\right)^{\alpha}\left(\frac{1}{1+\theta}\right)^{k}\frac{\Gamma(k+\alpha)}{k!\Gamma(\alpha)}
\end{align*}
We recognize this distribution as a negative binomial with parameters $p = \frac{\theta}{1+\theta}$ and $r = \alpha$

\end{description}	
\end{document} 


