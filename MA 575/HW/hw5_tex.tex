
\documentclass[12pt]{article}  %What kind of document (article) and what size
\usepackage[document]{ragged2e}


%Packages to load which give you useful commands
\usepackage{graphicx}
\usepackage{amssymb, amsmath, amsthm}
\usepackage{fancyhdr}
\usepackage[linguistics]{forest}
\usepackage{enumerate}
\usepackage[margin=0.75in]{geometry} 
\pagestyle{fancy}
\fancyhf{}
\lhead{MA 575: HW2}
\rhead{Benjamin Draves}


\renewcommand{\headrulewidth}{.4pt}
\renewcommand{\footrulewidth}{0.4pt}

%Sets the margins

%\textwidth = 8 in
%\textheight = 9.5 in

\topmargin = -0.4 in
%\headheight = 0.0 in t
%\headsep = .3 in
\parskip = 0.2in
%\parindent = 0.0in

%%%%%%%%%%new commands%%%%%%%%%%%%
\newcommand{\N}{{\mathbb{N}}}
\newcommand{\Z}{{\mathbb{Z}}}
\newcommand{\R}{{\mathbb{R}}}
\newcommand{\Q}{{\mathbb{Q}}}
\newcommand{\e}{{\epsilon}}
\newcommand{\del}{{\delta}}
\newcommand{\m}{{\mid}}
\newcommand{\infsum}{{\sum_{n=1}^\infty}}
\newcommand{\la}{{\langle}}
\newcommand{\ra}{{\rangle}}
\newcommand{\E}{{\mathbb{E}}}
\newcommand{\V}{{\mathbb{V}}}
\newcommand{\bb}{{\boldsymbol{\beta}}}

%defines a few theorem-type environments
\newtheorem{theorem}{Theorem}
\newtheorem{corollary}[theorem]{Corollary}
\newtheorem{definition}{Definition}
\newtheorem{lemma}[theorem]{Lemma}
%%%%% End of preamble %%%%%


\begin{document}

\textbf{Exercise 5.4}

Let $X = \begin{bmatrix}
1 & x_{11} & x_{21} & \dots & x_{p1}\\
1 & x_{12} & x_{22} & \dots & x_{p2}\\
\vdots & \vdots & \vdots & \ddots & \vdots\\
1 & x_{1n} & x_{2n}& \dots & x_{pn}\\
\end{bmatrix}$ 
Then we have 
\begin{align*}X^{T}X &= 
\begin{bmatrix}
1 & 1 & 1 & \dots & 1\\
x_{11} & x_{12} & x_{13} & \dots & x_{1n}\\
\vdots & \vdots & \vdots & \ddots & \vdots\\
x_{p1} & x_{p2} & x_{p3}& \dots & x_{pn}\\
\end{bmatrix}
\begin{bmatrix}
1 & x_{11} & x_{21} & \dots & x_{p1}\\
1 & x_{12} & x_{22} & \dots & x_{p2}\\
\vdots & \vdots & \vdots & \ddots & \vdots\\
1 & x_{1n} & x_{2n}& \dots & x_{pn}\\
\end{bmatrix} \\
&=\begin{bmatrix}
n & \sum_{i=1}^{n}x_{1i} & \sum_{i=1}^{n}x_{2i} & \dots & \sum_{i=1}^{n}x_{pi}\\
\sum_{i=1}^{n}x_{1i} & \sum_{i=1}^{n}x_{1i}^2 & \sum_{i=1}^{n}x_{1i}x_{2i}& \dots & \sum_{i=1}^{n}x_{1i}x_{pi}\\
\vdots & \vdots & \vdots & \ddots & \vdots \\
\sum_{i=1}^{n}x_{pi} & \sum_{i=1}^{n}x_{pi}x_{1i} & \sum_{i=1}^{n}x_{pi}x_{2i} & \dots & \sum_{i=1}^{n}x_{pi}^2\\
\end{bmatrix} 
\end{align*}

From this, let $A_{11} = [n]$ be the $1\times1$ matrix, $A_{12} = \big[\sum_{i=1}^{n}x_{1i} \hspace{1em} \sum_{i=1}^{n}x_{2i} \hspace{1em} \dots \hspace{1em}\sum_{i=1}^{n}x_{pi}\big]$, and let $A_{22} = (\sum_{i=1}^{n}x_{ki}x_{ji})_{1\leq k,j\leq p}$. 

First note that

\begin{align*}\mathcal{X}^{T}\mathcal{X} &= 
\begin{bmatrix}
(x_{11} - \overline{x}_1) & (x_{21}-\overline{x}_1) & \dots & (x_{n1}-\overline{x}_1)\\
(x_{12} - \overline{x}_2) & (x_{22}-\overline{x}_2) & \dots & (x_{n2}-\overline{x}_2)\\
\vdots & \vdots & \ddots & \vdots \\
(x_{1p} - \overline{x}_p) & (x_{2p}-\overline{x}_p) & \dots & (x_{np}-\overline{x}_p)\\
\end{bmatrix}
\begin{bmatrix}
(x_{11} - \overline{x}_1) & (x_{12}-\overline{x}_2) & \dots & (x_{1p}-\overline{x}_1)\\
(x_{21} - \overline{x}_1) & (x_{22}-\overline{x}_2) & \dots & (x_{2p}-\overline{x}_2)\\
\vdots & \vdots & \ddots & \vdots \\
(x_{n1} - \overline{x}_1) & (x_{n2}-\overline{x}_2) & \dots & (x_{np}-\overline{x}_p)\\
\end{bmatrix} \\
&=\begin{bmatrix}
\sum_{i=1}^{n}(x_{i1}-\overline{x}_1)^2 & \sum_{i=1}^{n}(x_{i1}-\overline{x}_{1})(x_{i2}-\overline{x}_2) & \dots & \sum_{i=1}^{n}(x_{i1}-\overline{x}_1)(x_{ip}-\overline{x}_{p})\\
\sum_{i=1}^{n}(x_{i1}-\overline{x}_1)(x_{i2}-\overline{x}_2) & \sum_{i=1}^{n}(x_{i2}-\overline{x}_2)^2 & \dots & \sum_{i=1}^{n}(x_{i2}-\overline{x}_2)(x_{ip}-\overline{x}_{p})\\
\vdots & \vdots & \ddots & \vdots\\
\sum_{i=1}^{n}(x_{i1}-\overline{x}_1)(x_{ip}-\overline{x}_p) & \sum_{i=1}^{n}(x_{i2}-\overline{x}_2)(x_{ip}-\overline{x}_p) & \dots & \sum_{i=1}^{n}(x_{ip}-\overline{x}_p)^2\\
\end{bmatrix} 
\end{align*}

Now, for \textit{any} $1\leq j\leq p$ and $1\leq k\leq p$

\begin{align*}
\sum_{i=1}^{n}(x_{ik}-\overline{x}_k)(x_{ij}-\overline{x}_{j}) &= \sum_{i=1}^{n}x_{ik}x_{ij} - \overline{x}_j\sum_{i=1}^{n}x_{ik} - \overline{x}_k\sum_{i=1}^{n}x_{ij} + n\overline{x}_j\overline{x}_k\\
&=\sum_{i=1}^{n}x_{ik}x_{ij} - 2n\overline{x}_j\overline{x}_k+n\overline{x}_j\overline{x}_k\\
&= \sum_{i=1}^{n}x_{ik}x_{ik} - n\overline{x}_{j}\overline{x}_k
\end{align*}

We will now show that our $A_{22}-A_{12}^{T}A_{11}^{-1}A_{12} = \mathcal{X}^{T}\mathcal{X}$. First note that $A_{12}^{T}A_{11}^{-1}A_{12} = \frac{1}{n}A_{12}^{T}A_{12}$. Moreover, the entries of $A_{12}^{T}A_{12}$ are given by $$A_{12}^{T}A_{12} = \big(\sum_{i=1}^{n}x_{ki}\sum_{i=1}^{n}x_{ji}\big)_{1\leq k,j\leq p} = (n^2\overline{x}_{k}\overline{x}_k\big)_{1\leq j,k\leq p}$$ Therefore we see $$A_{12}^{T}A_{11}^{-1}A_{12} = (n\overline{x}_k\overline{x}_j)_{1\leq k,j\leq p}$$
Combining this result with the definition of $A_{22} = (\sum_{i=1}^{n}x_{ki}x_{ji})_{1\leq k,j\leq p}$. Thus we see 

$$A_{22} - A_{12}^{T}A_{11}^{-1}A_{12} = \Big(\sum_{i=1}^{n}x_{ki}x_{ji} - n\overline{x}_k\overline{x}_k\Big)_{1\leq k,j\leq p} = \mathcal{X}^{T}\mathcal{X}$$

Now, notice that $A_{11}^{-1}A_{12} = (\frac{1}{n}\sum_{i=1}^{n}x_{ik})1\leq k\leq p = \mathbf{\overline{x}}^{T}$ and $A_{11}^{-1}A_{12}^{T} = (\frac{1}{n}\sum_{i=1}^{n}x_{ik})1\leq k\leq p = \mathbf{\overline{x}}$. Having shown these relationshops hold we have 

\begin{align*}
(\mathbf{X}^{T}\mathbf{X})^{-1} = \begin{bmatrix}
\frac{1}{n} +(\mathbf{\overline{x}}^{T})(\mathcal{X}^{T}\mathcal{X})^{-1}\mathbf{\overline{x}} & -\mathbf{\overline{x}}(\mathcal{X}^{T}\mathcal{X})^{-1}\mathbf{\overline{x}}\\
-(\mathcal{X}^{T}\mathcal{X})^{-1}\mathbf{\overline{x}} & (\mathcal{X}^{T}\mathcal{X})^{-1}\\
\end{bmatrix}
\end{align*}







\end{document}
