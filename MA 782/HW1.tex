%%%%% Beginning of preamble %%%%%

\documentclass[12pt]{article}  %What kind of document (article) and what size
\usepackage[document]{ragged2e}

%Packages to load which give you useful commands
\usepackage{graphicx}
\usepackage{amssymb, amsmath, amsthm}
\usepackage{fancyhdr}
\usepackage[linguistics]{forest}
\usepackage{enumerate}
\usepackage[margin=1in]{geometry} 
\pagestyle{fancy}
\fancyhf{}
\lhead{MA 782: HW1, \today}
\rhead{Benjamin Draves}

\renewcommand{\headrulewidth}{.4pt}
\renewcommand{\footrulewidth}{0.4pt}

\topmargin = -0.4 in
%\headheight = 0.0 in t
%\headsep = .3 in
\parskip = 0.2in
%\parindent = 0.0in

%%%%%%%%%%new commands%%%%%%%%%%%%
\newcommand{\N}{{\mathbb{N}}}
\newcommand{\Z}{{\mathbb{Z}}}
\newcommand{\R}{{\mathbb{R}}}
\newcommand{\Q}{{\mathbb{Q}}}
\newcommand{\e}{{\epsilon}}
\newcommand{\del}{{\delta}}
\newcommand{\m}{{\mid}}
\newcommand{\nsum}{{\sum_{i=1}^n}}
\newcommand{\la}{{\langle}}
\newcommand{\ra}{{\rangle}}
\newcommand{\E}{{\mathbb{E}}}
\newcommand{\V}{{\text{Var}}}
\newcommand{\prob}{{\mathbb{P}}}
\newcommand{\ind}{{\mathbf{1}}}
%defines a few theorem-type environments
\newtheorem{theorem}{Theorem}
\newtheorem{corollary}[theorem]{Corollary}
\newtheorem{definition}{Definition}
\newtheorem{lemma}[theorem]{Lemma}
%%%%% End of preamble %%%%%

\begin{document}
\begin{enumerate}
\item (6.3)
\begin{enumerate}
	\item First let $T_1$ and $T_2$ be UMP tests for corresponding to $\alpha_1$ and $\alpha_2$ respectively. Then for the sake of contradiction, assume that $c(\alpha_1)<c(\alpha_2)$. By Neyman-Person our two tests are given by the following 
	\[T_1 = \begin{cases}
	1 & f_1(x) > c(\alpha_1)f_0(x)\\
	\gamma_1 & f_1(x) = c(\alpha_1)f_0(x)\\
	0 & f_1(x) < c(\alpha_1)f_0(x)\\
	\end{cases}\hspace{2em}
	T_2 = \begin{cases}
	1 & f_1(x) > c(\alpha_2)f_0(x)\\
	\gamma_1 & f_1(x) = c(\alpha_2)f_0(x)\\
	0 & f_1(x) < c(\alpha_2)f_0(x)\\
	\end{cases}
	\]
	Define $LR(x) = \frac{f_1(x)}{f_0(x)}$. We now argue by cases
	\begin{itemize}
		\item $LR(x)<c(\alpha_1)$: In this case $T_1(x) = 0$ and $T_2(x)=$ so $T_1(x)\geq T_2(x)$
		\item $LR(x)=c(\alpha_1)$: In this case $T_1(x) = \gamma_1$ and $T_2(x)=0$ so $T_1(x)\geq T_2(x)$
		\item $c(\alpha_1)<LR(x)<c(\alpha_2)$: In this case $T_1(x) = 1$ and $T_2(x)=0$ so $T_1(x)\geq T_2(x)$
		\item $LR(x)=c(\alpha_2)$: In this case $T_1(x) = 1$ and $T_2(x)=\gamma_2$ so $T_1(x)\geq T_2(x)$
		\item $c(\alpha_2)<LR(x)$: In this case $T_1(x) = 1$ and $T_2(x)=1$ so $T_1(x)\geq T_2(x)$
	\end{itemize}
	In any case we note that $T_1(x)\geq T_2(x)$. Hence $\E_0(T_1) \geq \E_0(T_2)$ but recall this, by definition, is just the type I error corresponding to $\alpha_i$. That is we see that $\E_0(T_1) = \alpha_1 \geq \alpha_2=\E_0(T_2)$. But recall $\alpha_1<\alpha_2$ so we have a contradtiction. Therefore, we conclude that $c(\alpha_1)\geq c(\alpha_2)$
	\item We look to show that $1 - \E_1(T_1)\geq1 - \E_1(T_2)$. Hence, it suffies to show that $\E_1[T_1]\leq \E_1[T_2]$. Well recall that $\E_1[T_*]$ is maximized at $T_* = T_2$ where $T_*$ are all tests such that $\E_0[T_*]\leq \alpha_2$. But $\E_0[T_1] = \alpha_1 <\alpha_2$. Hence $\E_1[T_1]\leq \E_1[T_2]$ by definition of UMP. 
\end{enumerate}	
\item (6.4) Let $T_*$ be UMP for size $\alpha$ with power $\beta<1$ for testing $H_0$ vs $H_1$. Define $T_{**} = 1 - T_*$. We look to show that this test is UMP for testing $H_1$ vs $H_0$ with size $1-\beta$. We must show two things (i) $T_{**}$ is of size $\beta$ and (ii) for any test $\tilde{T}_{**}$ with $\E_1(\tilde{T}_{**})\leq \beta$ we have that $\E_0[\tilde{T}_{**}]\leq \E_0[T_{**}]$. 

\begin{enumerate}[(i)]
\item First note that $\E_1[T_{**}] = 1-\E_1[T_{*}]$. Note this is just the power of $T_*$ when $H_1$ is true which we defined to be $\beta$. Hence  $\E_1[T_{**}] = 1-\E_1[T_{*}] = 1-\beta$. Hence $T_{**}$ is of size $1-\beta$.
\item Let $\tilde{T}_{**}$ be a test with $\E_1[\tilde{T}_{**}]\leq 1 - \beta$. Recall that $T_{*}$ is UMP of level $\alpha$ for $H_0$ vs $H_1$. By the contrapositive, we have that for any test $\delta$ with $\E_1[\delta]>\E_1[T_*]$ implies $\E_0[\delta]>\E_0[T_*]$. But notice that we have $\E_1[1-\tilde{T}_{**}]\geq \beta = \E_1[T_*]$ so by the contrapositive $\E_0[1-\tilde{T}_{**}]\geq \alpha$. Upon rearranging, we have that $\E_0[\tilde{T}_{**}]\leq 1 - \alpha = 1-\E_0[T_*] = \E_0[1-T_*]$. 
\end{enumerate}
Therefore, $1-T_{*}$ is UMP for $H_1$ vs $H_0$ with level $1-\beta$. 

\item (6.7) Suppose that $\phi_*\in\mathcal{T}_0$ is given by equation $(6.9)$. We derive a useful result for later in the proof. 
\begin{align*}
\int(\phi_* - \phi)\sum_{i=1}^mc_if_id\nu&= \int\sum_{i=1}^{m}c_if_i\phi_*d\nu - \int\sum_{i=1}^{m}c_if_i\phi d\nu\\
&= \sum_{i=1}^mc_i\left\{\int \phi_*f_id\nu - \int\phi f_id\nu\right\}\\
&=\sum_{i=1}^mc_i \left\{0-0\right\}\\
&= 0
\end{align*}
Where the thrid line justified by the definition of $\mathcal{T}_0$. Now, define the following quantity 
$$g(x)h(x) = (\phi_* - \phi)(f_{m+1} - \sum_{i=1}^m c_if_i)$$ Now notice by defintion of $\phi_*$ if $h(x)>0$ then $g(x)>0$ and $g(x)h(x)\geq 0$. Moroever, if $h(x)<0$, then $g(x)<0$ as $\phi_* = 0$ and $0\leq\phi\leq 1$. In either cases we have that $$(\phi_* - \phi)(f_{m+1} - \sum_{i=1}^m c_if_i)\geq 0$$ Using this, we can write 
\begin{align*}
(\phi_* - \phi)(f_{m+1} - \sum_{i=1}^m c_if_i)&\geq 0\\
(\phi_* - \phi)f_{m+1}&\geq (\phi_* - \phi)\sum_{i=1}^m c_if_i\\
\int (\phi_* - \phi)f_{m+1}d\nu&\geq \int(\phi_* - \phi)\sum_{i=1}^m c_if_id\nu\\
\int \phi_*f_{m+1}d\nu - \int \phi f_{m+1}d\nu&\geq 0\\
\int \phi_*f_{m+1}d\nu  &\geq \int \phi f_{m+1}d\nu\\
\end{align*}
Hence, $\phi_*$ maximizes this quantity over all possible $\phi\in\mathcal{T}_0$.

Now, consider $\phi\in\mathcal{T}$. Then, by the same argument as above, we have that 
\begin{align*}
\int(\phi_* - \phi)f_{m+1}d\nu &\geq \int(\phi_* - \phi)\sum_{i = 1}^m c_if_id\nu\\
\int(\phi_* - \phi)f_{m+1}d\nu &\geq \sum_{i = 1}^m c_i\left\{\int\phi_*f_id\nu - \int\phi f_id\nu\right\}\\
\int(\phi_* - \phi)f_{m+1}d\nu &\geq \sum_{i = 1}^m c_i\left\{t_i - \int\phi f_id\nu\right\}\\
\end{align*}
Recall, by definition of $\mathcal{T}$ that $s_i:=\int\phi f_id\nu\leq t_i$. Therefore, along with the assumption that $c_i\geq 0$ we have 
$$\sum_{i = 1}^m c_i\left\{t_i - \int\phi f_id\nu\right\}\geq \sum_{i = 1}^m c_i\left\{t_i - s_i\right\}\geq 0$$ Thus 
\begin{align*}
\int(\phi_* - \phi)f_{m+1}d\nu &\geq 0\\
\int\phi_*f_{m+1}d\nu &\geq \int\phi_*f_{m+1}d\nu\\
\end{align*}
Therefore, if $c_i\geq 0$ for all $i$ then $\phi_*$ maximizes this quantity over all $\phi\in\mathcal{T}$. 

\end{enumerate}	
\end{document} 


