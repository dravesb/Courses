
%%%%% Beginning of preamble %%%%%

\documentclass[12pt]{article}  %What kind of document (article) and what size
\usepackage[document]{ragged2e}

%Packages to load which give you useful commands
\usepackage{graphicx}
\usepackage{amssymb, amsmath, amsthm}
\usepackage{fancyhdr}
\usepackage[linguistics]{forest}
\usepackage{enumerate}
\usepackage[margin=1in]{geometry} 
\pagestyle{fancy}
\fancyhf{}
\lhead{MA 780: HW5, \today}
\rhead{Benjamin Draves}

\renewcommand{\headrulewidth}{.4pt}
\renewcommand{\footrulewidth}{0.4pt}

\topmargin = -0.4 in
%\headheight = 0.0 in t
%\headsep = .3 in
\parskip = 0.2in
%\parindent = 0.0in

%%%%%%%%%%new commands%%%%%%%%%%%%
\newcommand{\N}{{\mathbb{N}}}
\newcommand{\Z}{{\mathbb{Z}}}
\newcommand{\R}{{\mathbb{R}}}
\newcommand{\Q}{{\mathbb{Q}}}
\newcommand{\e}{{\epsilon}}
\newcommand{\del}{{\delta}}
\newcommand{\m}{{\mid}}
\newcommand{\nsum}{{\sum_{i=1}^n}}
\newcommand{\la}{{\langle}}
\newcommand{\ra}{{\rangle}}
\newcommand{\E}{{\mathbb{E}}}
\newcommand{\V}{{\text{Var}}}
\newcommand{\prob}{{\mathbb{P}}}
\newcommand{\ind}{{\mathbf{1}}}
%defines a few theorem-type environments
\newtheorem{theorem}{Theorem}
\newtheorem{corollary}[theorem]{Corollary}
\newtheorem{definition}{Definition}
\newtheorem{lemma}[theorem]{Lemma}
%%%%% End of preamble %%%%%

\begin{document}
\begin{enumerate}
\item 
\begin{enumerate}
\item Using the fact that $h:\R\to[0,1]$ we have 
\begin{align*}
|f_h| &= \Big|e^{x^2/2}\int_{-\infty}^x \left(h(y)-\E(h(N))\right)e^{-y^2/2}dy\Big|\\
&\leq e^{x^2/2}\int_{-\infty}^x \Big|h(y)-\E(h(N))\Big|e^{-y^2/2}dy\\
&\leq e^{x^2/2}\int_{-\infty}^x 1\cdot e^{-y^2/2}dy\\
&\leq e^{x^2/2}\int_{-\infty}^x e^{-y^2/2}dy\\
\end{align*}
In a similar fashion we also have
\begin{align*}
|f_h| &= \Big|e^{x^2/2}\int_{-\infty}^x \left(h(y)-\E(h(N))\right)e^{-y^2/2}dy\Big|\\
&= \Big|e^{x^2/2}\int_{-\infty}^{\infty} \left(h(y)-\E(h(N))\right)e^{-y^2/2}dy - e^{x^2/2}\int_{x}^{\infty} \left(h(y)-\E(h(N))\right)e^{-y^2/2}dy\Big|\\
&\leq \Big|e^{x^2/2}\int_{-\infty}^{\infty} \left(h(y)-\E(h(N))\right)e^{-y^2/2}dy\Big| + \Big|e^{x^2/2}\int_{x}^{\infty} \left(h(y)-\E(h(N))\right)e^{-y^2/2}dy\Big|\\
&\leq \Big|e^{x^2/2}\left(\int_{-\infty}^{\infty}h(y)e^{-y^2/2}dy-\E(h(N))\int_{-\infty}^{\infty}e^{-y^2/2}dy\right)\Big| \\&+ \Big|e^{x^2/2}\int_{x}^{\infty} \left(h(y)-\E(h(N))\right)e^{-y^2/2}dy\Big|\\
&\leq \Big|e^{x^2/2}\left(\sqrt{2\pi}\E(h(N))-\sqrt{2\pi}\E(h(N))\right)\Big| + e^{x^2/2}\int_{x}^{\infty} \Big|h(y)-\E(h(N))\Big|e^{-y^2/2}dy\\
&\leq e^{x^2/2}\int_{x}^{\infty} e^{-y^2/2}dy\\
\end{align*}
Therefore we see that $$|f_h|\leq e^{x^2/2}\min\left\{\int_{-\infty}^xe^{-y^2/2}dy,\int_x^{-\infty}e^{-y^2/2}dy,\right\}$$
\item asdf
\item ???
\item By Stein's equation we know that $f_h'(x) = xf_h(x) + h(x)-\E[h(N)]$
\end{enumerate}	
\item 
\begin{enumerate}
\item 
\begin{align*}
f_z(x) &= e^{x^2/2}\int_{-\infty}^x\left(\ind_{(-\infty, z]}(y)-\E[\ind_{(-\infty, z]}(N)]\right)e^{-y^2/2}dy\\
&= e^{x^2/2}\left(\frac{\sqrt{2\pi}}{\sqrt{2\pi}}\int_{-\infty}^x\ind_{(-\infty, z]}(y)e^{-y^2/2}dy - \prob(N\leq z)\frac{\sqrt{2\pi}}{\sqrt{2\pi}}\int_{-\infty}^{x}e^{-y^2/2}dy\right)\\
&= e^{x^2/2}\left(\frac{\sqrt{2\pi}}{\sqrt{2\pi}}\int_{-\infty}^x\ind_{(-\infty, z]}(y)e^{-y^2/2}dy - \Phi(z)\sqrt{2\pi}\Phi(x)\right)
\end{align*}
Now, the first integral depends on $x\leq z$ or $x\geq z$. That is the integral's top limit will be given by $\min\{x,z\}$. Hence we can write $f_z$ in general as 
\[f_z(x) = 
\begin{cases}
\sqrt{2\pi}e^{x^2/2}\Phi(x)(1-\Phi(z)) & z\geq x\\
\sqrt{2\pi}e^{x^2/2}\Phi(z)(1-\Phi(x)) & z\leq x 
\end{cases}\]
\item Recall the useful fact that $\Phi(-z) = 1-\Phi(z)$ due to the symmetry of $\Phi(\cdot)$. With this fact we have 
\[f_{-z}(-x) = 
\begin{cases}
\sqrt{2\pi}e^{x^2/2}\Phi(-x)(1-\Phi(-z)) & -z\geq -x\\
\sqrt{2\pi}e^{x^2/2}\Phi(-z)(1-\Phi(-x)) & -z\leq -x 
\end{cases}\]
\[=
\begin{cases}
\sqrt{2\pi}e^{x^2/2}(1-\Phi(x))(\Phi(z)) & z\leq x\\
\sqrt{2\pi}e^{x^2/2}(1-\Phi(z))(\Phi(x)) & z\geq x 
\end{cases} = f_z(x)\] 
Here we may assume without loss of generality that $z\geq 0$. 
\item Now, we take the derivative of $xf_z(x)$ in an attempt to show the function is increasing in $x$. Let $\varphi$ be the density of a standard normal random variable. First we calculate the derivative of $f_h$. 

\begin{align*}
f_h'(x) &= \begin{cases}
\sqrt{2\pi}xe^{x^2/2}(1-\Phi(x))(\Phi(z)) -\sqrt{2\pi}e^{x^2/2}\varphi(x)\Phi(z) & z\leq x \\ 
\sqrt{2\pi}xe^{x^2/2}(1-\Phi(z))(\Phi(x)) +\sqrt{2\pi}e^{x^2/2}(1-\Phi(z))\varphi(x) & z\geq x \\
\end{cases}\\
&= \begin{cases}
\sqrt{2\pi}e^{x^2/2}\Phi(z)\left[x(1-\Phi(x)) -\varphi(x)\right]& z\leq x \\ 
\sqrt{2\pi}e^{x^2/2}(1-\Phi(z))\left[x\Phi(x) + \varphi(x)\right] & z\geq x \\
\end{cases}
&= \begin{cases}
xf(x)& z\leq x \\ 
& z\geq x \\
\end{cases}
\end{align*}
Therefore, in general 
\begin{align*}
\frac{d}{dx}[xf_h(x)] = f_h(x) + xf'_h(x)
\end{align*}

\end{enumerate}	

\end{enumerate}	
\end{document} 


